\documentclass{article}
% ============================================================
% PACKAGES — Empirical Template
% (Empirical ML / Applied Computer Science)
% ============================================================

% --- Math ---
\usepackage{amsfonts}
\usepackage{amsmath}
\usepackage{amsthm}     % \newtheorem, \theoremstyle
\usepackage{mathtools}  % \coloneqq, \DeclarePairedDelimiter, etc.

% --- Layout & Typography ---
\usepackage[margin=1in]{geometry}
\usepackage{microtype}
\usepackage{setspace}

% --- Tables ---
\usepackage{booktabs}    % \toprule, \midrule, \bottomrule
\usepackage{tabularx}    % Auto-width X columns
\usepackage{multirow}    % \multirow{n}{*}{text} — row-spanning cells
\usepackage{colortbl}    % \rowcolor{}, \cellcolor{} — highlighted rows/cells

% --- Figures ---
\usepackage{graphicx}
\usepackage{subcaption}  % subfigures with independent captions
\usepackage{wrapfig}     % \begin{wrapfigure}{r}{0.4\linewidth}

% --- Plots ---
\usepackage{pgfplots}
\pgfplotsset{compat=1.18}
\usepackage{pgfplotstable}   % Load data from .csv / .dat files

% --- Code snippets ---
\usepackage{listings}
\usepackage[ruled,vlined,linesnumbered]{algorithm2e}

% --- Numbers & Units ---
\usepackage{siunitx}
% Usage: \num{1234567}  ->  1,234,567
%        \SI{3.14}{GHz}
%        \SI{90.2}{\percent}
\sisetup{group-separator={,}, group-minimum-digits=4}

% --- Color ---
\usepackage{xcolor}
\definecolor{better}{RGB}{0,128,0}     % green — improvement over baseline
\definecolor{worse}{RGB}{180,0,0}      % red   — regression
\definecolor{rowhl}{RGB}{235,245,255}  % light blue — highlighted table row

% --- Drawing ---
\usepackage{tikz}
\usetikzlibrary{positioning, fit, decorations.pathreplacing, arrows.meta}

% --- References (hyperref second-to-last, cleveref last) ---
\usepackage[numbers]{natbib}  % \citep{key}, \citet{key}
\usepackage{hyperref}
\usepackage{cleveref}         % \cref{fig:x} -> "Figure 1"; load AFTER hyperref

% ============================================================
% LISTINGS STYLE — Python / shell code snippets
% ============================================================
\lstset{
  language=Python,
  basicstyle=\ttfamily\small,
  keywordstyle=\color{blue}\bfseries,
  commentstyle=\color{gray}\itshape,
  stringstyle=\color{orange!80!black},
  showstringspaces=false,
  breaklines=true,
  frame=single,
  numbers=left,
  numberstyle=\tiny\color{gray},
  xleftmargin=1.5em,
}

% ============================================================
% THEOREM-LIKE ENVIRONMENTS
% (lighter set than theory template — empirical papers rarely need proofs)
% ============================================================

\newtheorem{observation}{Observation}
\newtheorem{proposition}{Proposition}
\newtheorem{claim}{Claim}

\theoremstyle{definition}
\newtheorem{definition}{Definition}

\theoremstyle{remark}
\newtheorem{remark}{Remark}

% ============================================================
% MATH MACROS
% ============================================================

% --- Number sets ---
\newcommand{\R}{\mathbb{R}}
\newcommand{\N}{\mathbb{N}}

% --- Probability ---
\newcommand{\E}{\mathbb{E}}
\newcommand{\Prob}{\mathbb{P}}

% --- Paired delimiters ---
\DeclarePairedDelimiter{\norm}{\lVert}{\rVert}
\DeclarePairedDelimiter{\abs}{\lvert}{\rvert}

% --- Common operators ---
\DeclareMathOperator*{\argmin}{arg\,min}
\DeclareMathOperator*{\argmax}{arg\,max}
\DeclareMathOperator{\softmax}{softmax}

% ============================================================
% UTILITY MACROS
% ============================================================

% Blue highlight for key ideas or equations
\newcommand{\highlight}[1]{\textcolor{blue}{#1}}

% Red bold TODO annotation
\newcommand{\TODO}[1]{\textcolor{red}{\textbf{TODO: #1}}}

% ---- Result comparison annotations ----
% Usage in prose: "improves by \better{+2.3\%} over the baseline"
\newcommand{\better}[1]{\textcolor{better}{$\uparrow$\,#1}}
\newcommand{\worse}[1]{\textcolor{worse}{$\downarrow$\,#1}}

% ---- Formatted names (consistent typography) ----
\newcommand{\dataset}[1]{\textsc{#1}}   % \dataset{ImageNet}  ->  IMAGENET (small-caps)
\newcommand{\model}[1]{\textsc{#1}}     % \model{ViT-B/16}
\newcommand{\baseline}[1]{\texttt{#1}}  % \baseline{Adam}     ->  monospace
\newcommand{\metric}[1]{\textit{#1}}    % \metric{Top-1 Acc.} ->  italic

% Bold best result in a table cell
\newcommand{\best}[1]{\textbf{#1}}

% State-of-the-art tag
\newcommand{\sota}{\textbf{SotA}}


% ============================================================
% PAPER METADATA — edit before writing
% ============================================================
\title{%
  \textbf{Paper Title} \\[0.4em]
}
\author{%
  Author One, Author Two \\[0.2em]
  Conference/Journal Name, Year \\[0.2em]
}
\date{\today}

% ============================================================
\begin{document}
% ============================================================

\maketitle

% Quick-reference header — fill in before writing the body
\noindent
\textbf{TL;DR:} One-sentence summary of the method and headline result. \\[0.2em]
\medskip\hrule\bigskip


% ============================================================
\section{Motivation \& Problem}
% ============================================================
% What practical or scientific gap does this paper address?
% Why does the existing approach fall short?
% What is the research question in concrete, measurable terms?
%
% Good prompts:
%   - What failure mode of prior methods motivates this work?
%   - What would a practitioner lose if this paper had never been written?


% ============================================================
\section{Method}
% ============================================================

\paragraph{Key idea.}
% One or two sentences: what is the single most important insight?
% Use \highlight{} for the central equation or design choice.

\paragraph{Architecture / algorithm.}
% Describe the model or algorithm clearly.
% An algorithm2e block or a TikZ architecture diagram goes well here.

% \begin{algorithm2e}[H]
%   \caption{\TODO{Algorithm name}}
%   \KwIn{\TODO{inputs}}
%   \KwOut{\TODO{outputs}}
%   \TODO{steps}
% \end{algorithm2e}

\paragraph{Training objective.}
% State the loss function.
% \[ \mathcal{L}(\theta) = \TODO{\ldots} \]

\paragraph{Implementation details.}
% Optimizer, learning rate schedule, batch size, hardware, wall-clock time.
% Key hyperparameters (and whether they are sensitive).

\begin{itemize}
  \item \textbf{Optimizer:} \TODO{e.g., \baseline{AdamW}, $\beta_1=0.9$, $\beta_2=0.999$}
  \item \textbf{LR schedule:} \TODO{e.g., cosine decay, warmup steps}
  \item \textbf{Hardware:} \TODO{e.g., 8$\times$ A100 80GB, training time}
\end{itemize}


% ============================================================
\section{Experimental Setup}
% ============================================================

\paragraph{Datasets.}
\begin{itemize}
  \item \dataset{DatasetName} — \TODO{modality, size, task, train/val/test splits}
\end{itemize}

\paragraph{Baselines.}
\begin{itemize}
  \item \model{ModelName} \citep{example2024} — \TODO{brief description; why it is a fair comparison}
\end{itemize}

\paragraph{Evaluation metrics.}
\begin{itemize}
  \item \metric{Metric Name} (\texttt{abbrev.}) — \TODO{what it measures; higher or lower is better}
\end{itemize}

\paragraph{Reproducibility.}
% Is code / data released? Are seeds fixed? Are error bars reported?
\TODO{Code: yes/no. Data: public/private. Seeds: fixed/unreported. Error bars: yes/no.}


% ============================================================
\section{Results}
% ============================================================
% Report and interpret the main numbers.
% Convention: \best{} around best result per column; \better{}/\worse{} for deltas.
% Use \rowcolor{rowhl} to highlight the proposed method's row.

\begin{table}[h]
  \centering
  \caption{\TODO{Task and dataset. What is being compared? Which split?}}
  \label{tab:main}
  \begin{tabular}{lcc}
    \toprule
    \textbf{Method} & \metric{Metric 1 $\uparrow$} & \metric{Metric 2 $\downarrow$} \\
    \midrule
    \model{Baseline 1} \citep{}    & 00.0 & 00.0 \\
    \model{Baseline 2} \citep{}    & 00.0 & 00.0 \\
    \midrule
    \rowcolor{rowhl}
    \model{This Paper}             & \best{00.0} & \best{00.0} \\
    \bottomrule
  \end{tabular}
\end{table}

\paragraph{Key takeaways.}
\begin{itemize}
  \item \TODO{Most important number from the main table / figure and what it means.}
  \item \TODO{Second-most important finding.}
\end{itemize}

% Add more figures or tables here as needed.
% \begin{figure}[h]
%   \centering
%   \includegraphics[width=0.8\linewidth]{figures/result.pdf}
%   \caption{\TODO{Caption.}}
%   \label{fig:result}
% \end{figure}


% ============================================================
\section{Ablations}
% ============================================================
% What does each component contribute?
% Which design choices matter most and which are incidental?
% Report the delta from removing / replacing each component.

\begin{itemize}
  \item \textbf{\TODO{Component A} removed:} \TODO{e.g., \worse{$-$3.1\%} on Metric 1 — large effect}
  \item \textbf{\TODO{Component B} replaced with simpler version:} \TODO{e.g., negligible change}
\end{itemize}

% Optionally add an ablation table:
% \begin{table}[h]
%   \centering
%   \caption{Ablation on \dataset{DatasetName}.}
%   \begin{tabular}{lc}
%     \toprule
%     \textbf{Variant} & \metric{Metric $\uparrow$} \\
%     \midrule
%     Full model               & \best{00.0} \\
%     w/o component A          & 00.0 \\
%     w/o component B          & 00.0 \\
%     \bottomrule
%   \end{tabular}
% \end{table}


% ============================================================
\section{Limitations \& Failure Modes}
% ============================================================
% Where does the method underperform, break, or rely on strong assumptions?
% What is the compute / data / annotation cost?
% Are there fairness, robustness, or distribution-shift concerns?

\begin{itemize}
  \item \TODO{e.g., Requires large-scale pretraining data; unclear how to apply in low-resource settings.}
  \item \TODO{e.g., Evaluation restricted to English; multilingual performance unknown.}
\end{itemize}


% ============================================================
\section{Open Questions \& Extensions}
% ============================================================
% What natural follow-up experiments would you run?
% What applications or domains could this transfer to?
% What would you do differently if you were writing this paper?

\begin{itemize}
  \item \TODO{}
\end{itemize}


% ============================================================
\section{Personal Notes}
% ============================================================
% Connections to your own research.
% Intuitions about why the method works.
% Things to remember or questions to ask the authors.


% ============================================================
% REFERENCES
% Add entries to refs.bib and cite with \citep{key} or \citet{key}.
% ============================================================
\bibliographystyle{plainnat}
\bibliography{refs}

\end{document}
