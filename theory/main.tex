\documentclass{article}
% ============================================================
% PACKAGES — Empirical Template
% (Empirical ML / Applied Computer Science)
% ============================================================

% --- Math ---
\usepackage{amsfonts}
\usepackage{amsmath}
\usepackage{amsthm}     % \newtheorem, \theoremstyle
\usepackage{mathtools}  % \coloneqq, \DeclarePairedDelimiter, etc.

% --- Layout & Typography ---
\usepackage[margin=1in]{geometry}
\usepackage{microtype}
\usepackage{setspace}

% --- Tables ---
\usepackage{booktabs}    % \toprule, \midrule, \bottomrule
\usepackage{tabularx}    % Auto-width X columns
\usepackage{multirow}    % \multirow{n}{*}{text} — row-spanning cells
\usepackage{colortbl}    % \rowcolor{}, \cellcolor{} — highlighted rows/cells

% --- Figures ---
\usepackage{graphicx}
\usepackage{subcaption}  % subfigures with independent captions
\usepackage{wrapfig}     % \begin{wrapfigure}{r}{0.4\linewidth}

% --- Plots ---
\usepackage{pgfplots}
\pgfplotsset{compat=1.18}
\usepackage{pgfplotstable}   % Load data from .csv / .dat files

% --- Code snippets ---
\usepackage{listings}
\usepackage[ruled,vlined,linesnumbered]{algorithm2e}

% --- Numbers & Units ---
\usepackage{siunitx}
% Usage: \num{1234567}  ->  1,234,567
%        \SI{3.14}{GHz}
%        \SI{90.2}{\percent}
\sisetup{group-separator={,}, group-minimum-digits=4}

% --- Color ---
\usepackage{xcolor}
\definecolor{better}{RGB}{0,128,0}     % green — improvement over baseline
\definecolor{worse}{RGB}{180,0,0}      % red   — regression
\definecolor{rowhl}{RGB}{235,245,255}  % light blue — highlighted table row

% --- Drawing ---
\usepackage{tikz}
\usetikzlibrary{positioning, fit, decorations.pathreplacing, arrows.meta}

% --- References (hyperref second-to-last, cleveref last) ---
\usepackage[numbers]{natbib}  % \citep{key}, \citet{key}
\usepackage{hyperref}
\usepackage{cleveref}         % \cref{fig:x} -> "Figure 1"; load AFTER hyperref

% ============================================================
% LISTINGS STYLE — Python / shell code snippets
% ============================================================
\lstset{
  language=Python,
  basicstyle=\ttfamily\small,
  keywordstyle=\color{blue}\bfseries,
  commentstyle=\color{gray}\itshape,
  stringstyle=\color{orange!80!black},
  showstringspaces=false,
  breaklines=true,
  frame=single,
  numbers=left,
  numberstyle=\tiny\color{gray},
  xleftmargin=1.5em,
}

% ============================================================
% THEOREM-LIKE ENVIRONMENTS
% (lighter set than theory template — empirical papers rarely need proofs)
% ============================================================

\newtheorem{observation}{Observation}
\newtheorem{proposition}{Proposition}
\newtheorem{claim}{Claim}

\theoremstyle{definition}
\newtheorem{definition}{Definition}

\theoremstyle{remark}
\newtheorem{remark}{Remark}

% ============================================================
% MATH MACROS
% ============================================================

% --- Number sets ---
\newcommand{\R}{\mathbb{R}}
\newcommand{\N}{\mathbb{N}}

% --- Probability ---
\newcommand{\E}{\mathbb{E}}
\newcommand{\Prob}{\mathbb{P}}

% --- Paired delimiters ---
\DeclarePairedDelimiter{\norm}{\lVert}{\rVert}
\DeclarePairedDelimiter{\abs}{\lvert}{\rvert}

% --- Common operators ---
\DeclareMathOperator*{\argmin}{arg\,min}
\DeclareMathOperator*{\argmax}{arg\,max}
\DeclareMathOperator{\softmax}{softmax}

% ============================================================
% UTILITY MACROS
% ============================================================

% Blue highlight for key ideas or equations
\newcommand{\highlight}[1]{\textcolor{blue}{#1}}

% Red bold TODO annotation
\newcommand{\TODO}[1]{\textcolor{red}{\textbf{TODO: #1}}}

% ---- Result comparison annotations ----
% Usage in prose: "improves by \better{+2.3\%} over the baseline"
\newcommand{\better}[1]{\textcolor{better}{$\uparrow$\,#1}}
\newcommand{\worse}[1]{\textcolor{worse}{$\downarrow$\,#1}}

% ---- Formatted names (consistent typography) ----
\newcommand{\dataset}[1]{\textsc{#1}}   % \dataset{ImageNet}  ->  IMAGENET (small-caps)
\newcommand{\model}[1]{\textsc{#1}}     % \model{ViT-B/16}
\newcommand{\baseline}[1]{\texttt{#1}}  % \baseline{Adam}     ->  monospace
\newcommand{\metric}[1]{\textit{#1}}    % \metric{Top-1 Acc.} ->  italic

% Bold best result in a table cell
\newcommand{\best}[1]{\textbf{#1}}

% State-of-the-art tag
\newcommand{\sota}{\textbf{SotA}}


% ============================================================
% PAPER METADATA — edit before writing
% ============================================================
\title{%
  \textbf{Paper Title} \\[0.4em]
}
\author{%
  Author One, Author Two \\[0.2em]
  Conference/Journal Name, Year \\[0.2em]
}
\date{\today}

% ============================================================
\begin{document}
% ============================================================

\maketitle

% Quick-reference header — fill in before writing the summary
\noindent
\textbf{TL;DR:} One-sentence summary of the core result and its significance. \\[0.2em]
\medskip\hrule\bigskip

% ============================================================
\section{Formal Setting}
% ============================================================
% Define every object precisely before stating results.
% Notation established here should be used consistently throughout.
%
% Questions to answer:
%   - What is the input space \cX, label/output space \cY?
%   - What is the hypothesis class \cH or function class \cF?
%   - What is the learning model (PAC, online, SQ, query complexity)?
%   - For geometry: what are the objects (polytopes, point sets)?
%   - What is the loss, the measure of success, the computational model?

\paragraph{Objects.}
% e.g., "Let \cH \subseteq \{0,1\}^{\cX} be a hypothesis class over \cX \subseteq \R^d."

\paragraph{Goal.}
% What quantity is being bounded or computed?
% e.g., "Find the smallest sample size $n$ such that an algorithm outputs
%         $h \in \cH$ with $\Risk(h) \le \eps$ with probability $\ge 1-\delta$."


% ============================================================
\section{Main Results}
% ============================================================
% State the central theorems with full quantifiers.
% Use \highlight{} for the key bound or conclusion.
% Always distinguish upper bounds from lower bounds.

\begin{theorem}[\TODO{paraphrase result title here}]
  \label{thm:main}
  \TODO{State theorem precisely, including all quantifiers and dependencies on parameters.}
\end{theorem}

\noindent
\highlight{Takeaway:} \TODO{One sentence explaining what this bound means concretely.}

\bigskip

% State the matching lower bound if the paper proves one:
\begin{theorem}[Lower bound]
  \label{thm:lb}
  \TODO{State lower bound. Note whether it matches the upper bound up to constants / log factors.}
\end{theorem}

% ---- Comparison table (uncomment and fill in) ----
% \begin{table}[h]
%   \centering
%   \caption{Comparison of upper bounds. $d$ = dimension, $n$ = sample size.}
%   \label{tab:comparison}
%   \begin{tabular}{llll}
%     \toprule
%     \textbf{Reference} & \textbf{Bound} & \textbf{Model} & \textbf{Assumption} \\
%     \midrule
%     \citet{prior1}   & $\Oh(d / \eps^2)$       & PAC & Realizable \\
%     \citet{prior2}   & $\Oh(d \log(1/\eps) / \eps)$ & PAC & Agnostic \\
%     \textbf{This paper} & $\highlight{\Oh(...)}$ & ...  & ... \\
%     \bottomrule
%   \end{tabular}
% \end{table}


% ============================================================
\section{Key Assumptions}
% ============================================================
% List assumptions explicitly, with one \begin{assumption}...\end{assumption} each.
% For each: (i) is it standard or non-standard? (ii) is it necessary or just convenient?

\begin{assumption}
  \label{asm:1}
  \TODO{e.g., Realizability: $\exists\, h^* \in \cH$ such that $\Risk(h^*) = 0$.}
\end{assumption}

\noindent\textit{Necessity.} \TODO{Is there a lower bound showing this assumption cannot be dropped?}

\begin{assumption}
  \label{asm:2}
  \TODO{e.g., Bounded hypothesis class: $\VC(\cH) = d < \infty$.}
\end{assumption}


% ============================================================
\section{Proof Techniques}
% ============================================================
% Describe the proof architecture at a high level.
% Name the tools used (VC theory, Rademacher, chaining, primal-dual,
% net arguments, dimension reduction, probabilistic method, ...).
% Quote the key intermediate lemma and sketch why it is true.

\paragraph{High-level strategy.}
% e.g., "Reduce to a 1D problem via a random projection argument,
%         then apply a net argument over an \eps-cover of ..."

\paragraph{Key intermediate result.}

\begin{lemma}
  \label{lem:key}
  \TODO{State the pivotal lemma that drives the main proof.}
\end{lemma}

\proofidea{\TODO{Informal argument for the lemma.}}

\paragraph{Completing the proof of \cref{thm:main}.}
% How does the key lemma imply the main theorem?


% ============================================================
\section{Comparison to Prior Work}
% ============================================================
% How does this result improve on, generalize, or differ from prior work?
% Is the bound tight? Does it match the lower bound?
% What was the bottleneck in the previous best approach?

\begin{itemize}
  \item \textbf{Improvement over \citet{}:} \TODO{what changes and by how much?}
  \item \textbf{New technique vs.\ prior art:} \TODO{what is the novel proof ingredient?}
\end{itemize}


% ============================================================
\section{Limitations \& Open Questions}
% ============================================================

\paragraph{Limitations.}
% Which assumptions are restrictive?
% Are there known settings where the result does not extend?

\begin{itemize}
  \item \TODO{e.g., Requires realizability — does not apply to the agnostic setting.}
\end{itemize}

\paragraph{Open questions.}

\begin{question}
  \label{q:1}
  \TODO{Most natural open problem left by this work.}
\end{question}

\begin{question}
  \label{q:2}
  \TODO{Second open question.}
\end{question}


% ============================================================
\section{Personal Notes}
% ============================================================
% Connections to your own research or other papers you know.
% Intuitions, analogies, or things you want to remember.
% Questions to ask if you meet the authors.


% ============================================================
% REFERENCES
% Add entries to refs.bib and cite with \citep{key} or \citet{key}.
% Example: "As shown by \citet{example2024}, ..."
% ============================================================
\bibliographystyle{plainnat}
\bibliography{refs}

\end{document}
