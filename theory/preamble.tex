% ============================================================
% PACKAGES — Theory Template
% (Learning Theory / Computational Geometry)
% ============================================================

% --- Math ---
\usepackage{amsfonts}   % \mathbb, \mathfrak fonts
\usepackage{amsmath}    % Core math environments: align, equation, gather, etc.
\usepackage{amssymb}    % Extra symbols: \triangleq, \square, \lesssim, etc.
\usepackage{amsthm}     % Theorem/lemma/definition environments
\usepackage{mathtools}  % Extends amsmath: \coloneqq, \DeclarePairedDelimiter, etc.

% --- Layout & Typography ---
\usepackage[margin=1in]{geometry}
\usepackage{microtype}
\usepackage{setspace}

% --- Tables ---
\usepackage{booktabs}   % \toprule, \midrule, \bottomrule
\usepackage{tabularx}   % Auto-width X columns

% --- Figures ---
\usepackage{graphicx}
\usepackage{subcaption}

% --- Drawing (geometric constructions, proof diagrams) ---
\usepackage{tikz}
\usetikzlibrary{%
  positioning, fit, calc, arrows.meta,
  decorations.pathreplacing,       % braces and annotations
  patterns, shapes.geometric,      % filled regions, polygons
  backgrounds                      % framed figures
}

% --- Algorithms ---
% Usage: \begin{algorithm2e}[H] \KwIn{...} \KwOut{...} ... \end{algorithm2e}
\usepackage[ruled,vlined,linesnumbered]{algorithm2e}

% --- Color ---
\usepackage{xcolor}

% --- References (hyperref second-to-last, cleveref last) ---
\usepackage[numbers]{natbib}  % \citep{key}, \citet{key}
\usepackage{hyperref}         % Clickable cross-refs and PDF metadata
\usepackage{cleveref}         % \cref{thm:main} -> "Theorem 1"; load AFTER hyperref

% ============================================================
% THEOREM ENVIRONMENTS
% Theorems, lemmas, and claims share one counter so that
% Theorem 1, Lemma 2, Claim 3 are never ambiguous.
% Definitions, examples, remarks have independent counters.
% ============================================================

\newtheorem{theorem}{Theorem}
\newtheorem{lemma}[theorem]{Lemma}
\newtheorem{claim}[theorem]{Claim}
\newtheorem{corollary}[theorem]{Corollary}
\newtheorem{proposition}[theorem]{Proposition}
\newtheorem{fact}[theorem]{Fact}
\newtheorem{conjecture}[theorem]{Conjecture}
\newtheorem{problem}{Problem}

\newtheorem{definition}{Definition}
\newtheorem{example}{Example}

\theoremstyle{remark}
\newtheorem{remark}{Remark}
\newtheorem{assumption}{Assumption}
\newtheorem{question}{Question}

% ============================================================
% MATH MACROS — General
% ============================================================

% --- Number sets ---
\newcommand{\R}{\mathbb{R}}
\newcommand{\N}{\mathbb{N}}
\newcommand{\Z}{\mathbb{Z}}
\newcommand{\Q}{\mathbb{Q}}
\newcommand{\C}{\mathbb{C}}

% --- Probability & Statistics ---
\newcommand{\E}{\mathbb{E}}             % Expectation: \E[X], \E_{x \sim p}[f(x)]
\newcommand{\Prob}{\mathbb{P}}          % Probability: \Prob(A)
\DeclareMathOperator{\Var}{Var}
\DeclareMathOperator{\Cov}{Cov}

% --- Paired delimiters (star variant auto-scales: \norm*{\frac{a}{b}}) ---
\DeclarePairedDelimiter{\norm}{\lVert}{\rVert}
\DeclarePairedDelimiter{\abs}{\lvert}{\rvert}
\DeclarePairedDelimiter{\inner}{\langle}{\rangle}
\DeclarePairedDelimiter{\ceil}{\lceil}{\rceil}
\DeclarePairedDelimiter{\floor}{\lfloor}{\rfloor}
\DeclarePairedDelimiter{\set}{\{}{\}}           % \set{x : f(x) > 0}
\DeclarePairedDelimiter{\card}{\lvert}{\rvert}  % \card{S} — cardinality

% --- Common operators ---
\DeclareMathOperator*{\argmin}{arg\,min}
\DeclareMathOperator*{\argmax}{arg\,max}
\DeclareMathOperator{\tr}{tr}
\DeclareMathOperator{\rank}{rank}
\DeclareMathOperator{\diag}{diag}
\DeclareMathOperator{\sign}{sign}
\DeclareMathOperator{\conv}{conv}   % Convex hull: \conv(S)
\DeclareMathOperator{\diam}{diam}   % Diameter: \diam(S)
\DeclareMathOperator{\vol}{vol}     % Volume: \vol(K)
\DeclareMathOperator{\proj}{proj}   % Projection: \proj_C(x)
\DeclareMathOperator{\dist}{dist}   % Distance: \dist(x, C)
\DeclareMathOperator{\spn}{span}    % Span

% ============================================================
% MATH MACROS — Learning Theory
% ============================================================

% --- Asymptotic notation ---
\newcommand{\Oh}{\mathcal{O}}               % Big-O:     \Oh(n \log n)
\newcommand{\oh}{o}                         % Little-o:  \oh(n)
\newcommand{\Og}{\Omega}                    % Big-Omega: \Og(n)
\newcommand{\Th}{\Theta}                    % Theta:     \Th(n)
\newcommand{\Oht}{\widetilde{\mathcal{O}}}  % Soft-O (hides log factors): \Oht(n)

% --- Sample / query complexity ---
\newcommand{\VC}{\mathrm{VC}}               % VC dimension:          \VC(\cH)
\DeclareMathOperator{\Rdim}{\mathfrak{R}}   % Rademacher complexity: \Rdim_n(\cH)
\newcommand{\cN}{\mathcal{N}}               % Covering number:       \cN(\cH,\eps,d)

% --- Spaces and classes ---
\newcommand{\cH}{\mathcal{H}}   % Hypothesis class
\newcommand{\cF}{\mathcal{F}}   % Function class
\newcommand{\cX}{\mathcal{X}}   % Input space
\newcommand{\cY}{\mathcal{Y}}   % Output space
\newcommand{\cD}{\mathcal{D}}   % Distribution
\newcommand{\cA}{\mathcal{A}}   % Algorithm
\newcommand{\cS}{\mathcal{S}}   % Sample
\newcommand{\cK}{\mathcal{K}}   % Convex body (geometry)

% --- Risk ---
\newcommand{\Risk}{\mathrm{R}}      % True risk:     \Risk(h)
\newcommand{\hRisk}{\hat{\mathrm{R}}} % Empirical risk: \hRisk_n(h)

% --- Information theory ---
\DeclareMathOperator{\KL}{KL}   % KL divergence:   \KL(p \| q)
\DeclareMathOperator{\MI}{I}    % Mutual information: \MI(X;Y)
\DeclareMathOperator{\Ent}{H}   % Entropy:           \Ent(X)

% --- Computational complexity classes ---
\newcommand{\Pcl}{\mathsf{P}}
\newcommand{\NP}{\mathsf{NP}}
\newcommand{\coNP}{\mathsf{coNP}}
\newcommand{\PSPACE}{\mathsf{PSPACE}}
\newcommand{\BPP}{\mathsf{BPP}}

% ============================================================
% MATH MACROS — Misc shorthands
% ============================================================

\newcommand{\eps}{\varepsilon}                              % \eps  (shorter than \varepsilon)
\newcommand{\defeq}{\coloneqq}                             % :=
\newcommand{\Ind}[1]{\mathbf{1}\!\left[#1\right]}          % Indicator: \Ind{x > 0}
\newcommand{\iid}{\overset{\mathrm{iid}}{\sim}}            % X_1,\ldots,X_n \iid \cD
% \lesssim is already provided by amssymb — no redefinition needed
\newcommand{\poly}{\mathrm{poly}}                          % polynomial: \poly(n)
\newcommand{\polylog}{\mathrm{polylog}}                    % polylogarithmic

% ============================================================
% UTILITY MACROS
% ============================================================

% Blue highlight for key equations or claims
\newcommand{\highlight}[1]{\textcolor{blue}{#1}}

% Red bold TODO annotation
\newcommand{\TODO}[1]{\textcolor{red}{\textbf{TODO: #1}}}

% Inline proof sketch paragraph (italic, no QED box)
% Usage: \proofidea{The key insight is a reduction to the 1D case via ...}
\newcommand{\proofidea}[1]{%
  \smallskip\noindent\textit{Proof idea.}\ #1\smallskip}
